%!TeX root=../index.tex

\section{Introduction}\label{sec:introduction}

% Short Intro

% Context

The trend away from licensed, on\-/premise enterprise software towards so called \gls{saas} has been going strong for the past decade and shows no sign of stopping.
This new paradigm---characterized by pay\-/per\-/use business models and provisioning on the provider's rather than the customer's infrastructure---also has a profound impact on the design of modern enterprise software.
\gls{saas} applications need to be always available, need to scale seamlessly according to changes in demand and have to minimize operating cost since it now directly affects the provider's bottom line.

These new requirements have paved the way for new trends in software architecture.
Modern applications are now commonly composited of multiple, integrated micro\-/services developed by mostly autonomous and independent teams.
Additionally, while the synchronous, HTTP\-/based mode of communication still dominates the user\-/system communication, asynchronous, message-based approaches have become popular for intra-application communication.

In systems that employ both of these patterns---micro\-/services and message\-/based communication---messages quickly exceed synchronous HTTP\-/calls in volume.
As a result, the correct use of messages by the system's individual components becomes paramount to the overall system's availability.
Schema technologies like Apache Avro\footnote{\url{https://avro.apache.org/}}, JSON Schema\footnote{\url{https://json-schema.org/}}, Apache Thrift\footnote{\url{https://thrift.apache.org/}} or Google's Protocol Buffers\footnote{\url{https://developers.google.com/protocol-buffers/}} can be used to establish a shared understanding about a message's payload between publishers and consumers.
But, how are schemas going to be distributed to message consumers?
And how can the communication contract represented by schemas be enforced to ensure that publishers will not break it?
Also, as with any \gls{api}, schemas have to evolve to match changes in the system.
How can such changes be facilitated while still enforcing the communication contract?

All of these questions belong to the domain of Schema Management and many projects rely on a dedicated software component to implement the requirements of Schema Management---a so\-/called Schema Registry.


% - Current trends in application development
%   - software as a service
%     - availability as top priority
%     - variable scaling
%     - low tco
%   - microservices
%     - autonomous teams
%     - distributed development
%   - message-oriented middleware
% - Schema Management

% Motivation

% - Literature that describes message-oriented systems seldomly make mention of schema management or schema registry applications
% - articles that include schema management in their design exclusively refer to the confluent schema registry as a solution
% - message-oriented systems become more popular and more companies become aware of the advantages of schema management, other schema registry solutions than the confluent schema registry have become available

% Goal

This paper aims to clarify the domain of schema management for newcomers from both academics and the private sector, so that its advantages become more accessible to a wider audience. To achieve this, I analyze the use case and the requirements that a schema registry addresses before I give an account of the solutions currently available.

% Structure
