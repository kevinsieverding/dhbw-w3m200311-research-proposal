% !TeX root = ../index.tex

\section{Related Work}\label{sec:related-work}

The first references to the use of schemas for inter-application communication in academic literature can be found in the early 2000s.
A considerable subset of the research conducted in this era has a strong focus on using schemas to not only achieve syntactic but also \emph{semantic} interoperability by trying to model a domain's semantics as machine-readable ontologies.
These efforts can be considered a part of the larger push towards the so\-/called \enquote{Semantic Web}---the vision for an internet of interlinked, machine-readable data---which mainly took place in that era \parencite{noauthor_semantic_nodate}.

\cite{dogac_semantically_2004}, for example, describe a way of extending web services in the travel industry with semantics by employing \gls{owl} besides \gls{xml} schemas that are already defined by an industry consortium.
Furthermore, \cite{crapo_semantically_2009} attempt to apply Semantic Web technologies in the context of creating a more intelligent power grid---termed the \enquote{Smart Grid}---which can be considered an early \gls{iot} use case. They also rely on \gls{owl} in that context.
Finally, \cite{heery_metadata_2003} describe their implementation of a registry for semantic schemas in \gls{owl} and \gls{rdf} formats.

As to research which does not concern itself with semantic modeling, \cite{duftler_web_2001} describes an interface modeling language based on \gls{wsdl} that is supposed to hide the details of underlying technologies such as SOAP.
Also, \cite{do_matching_2007} present a generic system for matching complex \gls{xml} schemas.

To summarize, early research around schemas often aims beyond the syntactic interoperability of applications as part of a larger push for the Semantic Web.
Yet, all of them have the underlying goal of improving the integration of independent and autonomous applications in a distributed system.
Additionally, \cites{duftler_web_2001}{dogac_semantically_2004}{li_semantic_2004}{crapo_semantically_2009} mention the exchange of messages, though none except for \cite{li_semantic_2004} do so in concert with \gls{mom}.

Coincidental with the Semantic Web's loss in popularity in the late 2000s, research concerning schemas dies down around that time.
A notable exception is \cite{ma_iip_2010}, which presents a generic event\-/based platform for Intelligent Transportation Systems.
In the course of laying out their design, they provide a definition of an event schema registry \parencite[see][p.~3]{ma_iip_2010}.

Not much later, \cite{kreps_kafka_2011} introduces a new open\-/source event streaming platform called Kafka---which can be considered a form of \gls{mom}. It also describes a schema registry for managing and distributing message schemas in Apache Avro format as a part of the solution.
Following Kafka's introduction to the Apache project and its subsequent rise in popularity for \gls{mom} use cases, mentions of a schema registry almost exclusively occur in research that describes various system designs using Apache Kafka.
\cites{muller_iot_2017}{radchenko_micro-workflows_2018}{ranjan_radar-base_2019} do so in the context of \gls{iot}, whereas \cite{g_b_high_2021} present a generic \gls{mom} that combines Apache Kafka, a schema registry and Apache Camel.
In addition to the above mentioned papers, the theses \cites{dessalegn_muruts_multi-tenant_2016}{auer_distributed_2017}{korhonen_using_2019} also describe system designs using Apache Kafka and a schema registry.

Of these works, \cites{muller_iot_2017}{radchenko_micro-workflows_2018}{dessalegn_muruts_multi-tenant_2016}{auer_distributed_2017}{korhonen_using_2019} explicitly name the schema registry that they use as the Confluent Schema Registry, which is the same registry that was initially presented in \cite{kreps_kafka_2011}.
\cites{ranjan_radar-base_2019}{g_b_high_2021} do not specify which schema registry they use, but it is safe to assume that it is the Confluent Schema Registry as well, since the former makes numerous references to the \enquote{Confluent Platform} and the latter's description of the registry's functionality matches closely with the Confluent Schema Registry.

To summarize, contemporary research which mentions schema management and schema management solutions in the form of a schema registry exclusively does so in the context of system designs using Apache Kafka.
Furthermore, the Confluent Schema Registry appears to be the de\-/facto standard for these use cases.

These findings raise the question of whether the relevance of schemas and schema management for contemporary information systems is really reduced to the Apache Kafka ecosystem as the progressively narrowing ambitions of research over the course of the past 20 years seems to indicate.
In the contemporary works referenced above, schema management is treated as a minor concern and not as main object of study, which begs the question whether this might constitute a gap in the research.
Additionally, all system designs in contemporary literature appear to default to the Confluent Schema Registry, which raises the question if there might exist alternatives which would be more suitable---at least for certain use cases.
